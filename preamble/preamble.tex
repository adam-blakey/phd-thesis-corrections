% Silence specific warnings.
\usepackage{silence}
\WarningsOff[gensymb,everypage,siunitx,todonotes]

% Most package includes.
%\usepackage[paper=a4paper, inner=3.8cm, outer=2.5cm, bottom=2.5cm, top=2.5cm]{geometry}
\usepackage[paper=a4paper, inner=2.5cm, outer=2.5cm, bottom=2.5cm, top=2.5cm]{geometry}
\setlength{\marginparwidth}{2cm}
\usepackage[english]{babel}
\usepackage[protrusion=true,expansion=true]{microtype}
\usepackage{amsmath,amsfonts,amsthm,amssymb}
%\usepackage[pdftex,draft]{graphicx}
\usepackage[pdftex]{graphicx}
%\usepackage[export]{adjustbox}
\usepackage[svgnames]{xcolor}
\usepackage[hang, small,labelfont=bf,up,textfont=it,up]{caption}
\usepackage{epstopdf}
\usepackage{booktabs}
\usepackage{fix-cm}
\usepackage{sobolev}
\usepackage[sorting=none,style=numeric-comp,url=false,maxbibnames=99]{biblatex}
\usepackage{url}
\usepackage{physics}
\usepackage{caption}
\usepackage[labelformat=simple]{subcaption}
\usepackage{bm}
\usepackage{siunitx}
\AtBeginDocument{\RenewCommandCopy\qty\SI}
\usepackage{enumitem}
\usepackage{calc}
\usepackage{tikz}
\usetikzlibrary{shadows.blur}
\usetikzlibrary{external}
\usepackage[loadshadowlibrary, obeyDraft]{todonotes}
%\usepackage[loadshadowlibrary]{todonotes}
\usepackage{gensymb}
\usepackage{csquotes}
\usepackage{epsfig}
\usepackage[linesnumbered]{algorithm2e}
\usepackage{tablefootnote}
\usepackage{setspace}
\usepackage{emptypage}
%\usepackage{mathpazo} % Font styles: https://tug.org/pracjourn/2006-1/hartke/hartke.pdf
\usepackage[some]{background}
\usepackage{dsfont}
\usepackage{floatrow}
\usepackage{changepage}
\usepackage{dirtytalk}
%\usepackage{lettrine}
\usepackage{floatpag}

\RestyleAlgo{ruled}

\addbibresource{./references-zotero.bib}

%% Custom colours.
\definecolor{UoNBlue1}{RGB}{0, 155, 189}
\definecolor{UoNBlue2}{RGB}{0, 125, 168}
\definecolor{UoNBlue3}{RGB}{0, 86, 151}
\definecolor{UoNBlue4}{RGB}{27, 42, 107}
\definecolor{UoNBlue5}{RGB}{25, 26, 79}
\definecolor{Green1}{HTML}{00ab61}
\definecolor{Green2}{HTML}{009756}
\definecolor{Green3}{HTML}{00844b}
\definecolor{Green4}{HTML}{007141}
\definecolor{Green5}{HTML}{005f36}
\definecolor{UoNOrange}{RGB}{243, 146, 0}
\definecolor{UoNRed}{RGB}{216, 15, 42}
\definecolor{UoNGreen}{RGB}{0, 95, 54}
\definecolor{UoNApple}{RGB}{147, 213, 0}
\definecolor{UoNBlue}{RGB}{0, 155, 193}
\definecolor{UoNPurple}{RGB}{121, 45, 133}

% Loaded after colours.
%\usepackage[colorlinks=true,pdfstartview=FitV,linkcolor=black,citecolor=bluegray,urlcolor=black,pdfpagelabels,naturalnames=true]{hyperref}
\usepackage{hyperref}

\definecolor{darkgray}{gray}{0.4}  
\definecolor{ddarkgray}{gray}{0.2} 
\definecolor{redgray}{rgb}{0.5,0.25,0.25}  
\definecolor{bluegray}{rgb}{0.25,0.25,0.5}

%% Custom sectioning.
\usepackage{titlesec}
\titleformat{name=\chapter,numberless}[display]{\normalfont\sffamily\huge\bfseries\color{UoNBlue1}}{}{0pt}{}
% Styles each chapter; e.g. "Chapter 3", or "Appendix B"
\makeatletter
\titleformat{\chapter}[display]{\normalfont\sffamily\huge\bfseries\color{UoNBlue1}}{}{0pt}{\@chapapp\,\thechapter\\ }
\makeatother
\titleformat{\section}[display]{\normalfont\sffamily\large\bfseries\color{UoNBlue2}}{}{0pt}{\thesection\ }
\titleformat{\subsection}[display]{\normalfont\sffamily\large\bfseries\color{UoNBlue3}}{}{0pt}{\thesubsection\ }
\titleformat{\subsubsection}[display]{\normalfont\sffamily\large\bfseries\color{UoNBlue4}}{}{0pt}{\thesubsubsection\ }

% \usepackage{sectsty}

% \chapterfont{
% 	\color{UoNBlue1}
% }
% \sectionfont{
% 	\color{UoNBlue2}
% }
% \subsectionfont{
% 	\color{UoNBlue3}
% }
% \subsubsectionfont{
% 	\color{UoNBlue4}
% }

% Table of Contents
\renewcommand*\contentsname{Contents}
\usepackage[titletoc]{appendix}
\usepackage{titletoc} % following is based on https://tex.stackexchange.com/questions/192786/lines-within-toc

%% Headers and footers.
\usepackage{fancyhdr}												
\pagestyle{fancy}												
\usepackage{lastpage}
\usepackage{etoolbox}
\patchcmd{\chapter}{plain}{empty}{}{}

% Header (empty).
\lhead{}
\chead{}
\rhead{}

% Footer (not empty).
%\lfoot{\iffloatpage{}{\footnotesize \texttt{\reportauthor} \textbullet \, \reporttitle}}
\lfoot{\footnotesize \texttt{\reportauthor} \textbullet \, \reporttitle}
\cfoot{}
%\rfoot{\iffloatpage{}{\footnotesize \thepage\ of \pageref{LastPage}}} % Put into main.tex and introduction.tex so it can be different on first few pages.
\renewcommand{\headrulewidth}{0.0pt}
%\renewcommand{\footrulewidth}{\iffloatpage{0.0pt}{0.4pt}}
\renewcommand{\footrulewidth}{0.4pt}

% Adds footer to chapter pages.
\assignpagestyle{\chapter}{fancy}

% Sets style for float pages.
\floatpagestyle{fancy}

%% Numbering down to the \subsection  level (but not \subsubsection in ToC).
\setcounter{tocdepth}{2}
\setcounter{secnumdepth}{3}

%  OBSLETE WITH THE LETTRINE PACKAGE 
%% Creating an initial of the very first character of the content
% \usepackage{lettrine}
% \newcommand{\initial}[1]{%
%      \lettrine[lines=3,lhang=0.3,nindent=0em]{
% 		\color{UoNBlue1}
% 		{\textrm{#1}}}{}}

%% Broken gradient.
\newcommand{\nablab}{\nabla_h}

%% Other DGFEM stuff.
\newcommand{\weakgrad}{\nabla_h}

%% Jumps and averages.
\newcommand{\jump}[1]{[\negthinspace[ #1 ]\negthinspace]}
\newcommand{\average}[1]{\{\negthickspace\{#1\}\negthickspace\}}
\newcommand{\jumpvec}[1]{\underline{\jump{#1}}}
\newcommand{\averagevec}[1]{\underline{\average{#1}}}

%% Dimensionless quantities.
\DeclareMathOperator\Rey{\mbox{\textit{Re}}}  % Reynold number
\DeclareMathOperator\Pec{\mbox{\textit{Pe}}}  % Péclet number
\DeclareMathOperator\Dam{\mbox{\textit{Dm}}}  % Damköhler number
\DeclareMathOperator\Dar{\mbox{\textit{Dr}}}  % Darcy number

%% Common sets
\newcommand{\RR}{\mathbb{R}}
\newcommand{\NN}{\mathbb{N}}
\newcommand{\CC}{\mathbb{C}}

%% Integration
%\newcommand*\diff{\mathop{}\!\m{d}}
%\newcommand{\diff}{\!\m{d}}
\newcommand{\diff}{~\mathrm{d}}

%% Bold for vectors
\renewcommand{\vec}[1]{\boldsymbol{\mathbf{#1}}}

%% Bold for matrices
%\def\mat#1{\underline{\underline{#1}}}
%\def\mat#1{\underline{#1}}
\def\mat#1{\boldsymbol{\mathbf{#1}}}

%% Title, author, and date metadata.
% \usepackage{titling}			% For custom titles
% \newcommand{\HorRule}{ 			% Creating a horizontal rule
% 	\color{UoNBlue1}			
%   	\rule{\linewidth}{1pt}
% }
% \pretitle{
% 	\includegraphics[width=5cm]{resources/uon-logo-black.png}
% 	\vspace{-30pt} \begin{flushleft} \HorRule 
% 	\fontsize{50}{50} \usefont{OT1}{phv}{b}{n} \color{UoNBlue1} \selectfont
% }
% \title{\reporttitle}
% \posttitle{
% 	\par\end{flushleft}\vskip 0.5em
% 	\LARGE \lineskip 0.5em \usefont{OT1}{phv}{b}{n} \color{UoNBlue3}
% 	\reportsubtitle
% 	\lineskip 1em
% }
% \preauthor{
% 	\begin{flushleft}
% 	\large \lineskip 0.5em 
% }
% \author{\usefont{OT1}{phv}{b}{sl}{\color{UoNBlue4}{\reportauthor}}\usefont{OT1}{phv}{r}{sl}{\color{black}{, supervised by: }}\newline\usefont{OT1}{phv}{b}{sl}{\color{UoNBlue5}{\reportsupervisors}}}
% \postauthor{
% 	\par\end{flushleft}\HorRule
% }
% \date{\reportdate}

% Starting index for section numbering.
\setcounter{section}{0}

% Brackets around subfigure references.
\renewcommand\thesubfigure{(\alph{subfigure})}

% Custom todo style.
\newcommand{\todoitemone}[1]{\tikzexternaldisable\todo[inline,textcolor=white,backgroundcolor=UoNRed,shadow]{\S\thesubsection~\textbf{TO-DO P1}: #1}\tikzexternalenable}
\newcommand{\todoitemtwo}[1]{\tikzexternaldisable\todo[inline,textcolor=white,backgroundcolor=Green3,shadow]{\S\thesubsection~\textbf{TO-DO P2}: #1}\tikzexternalenable}
\newcommand{\todoitemthree}[1]{\tikzexternaldisable\todo[inline,textcolor=white,backgroundcolor=UoNBlue3,shadow]{\S\thesubsection~\textbf{TO-DO P3}: #1}\tikzexternalenable}

% Exponentials.
%\renewcommand{\exp}[1]{e^{#1}}

% Imaginary unit.
\newcommand{\iu}{{i\mkern1mu}}

% Nomenclature.
% \usepackage{nomencl}
% \makenomenclature

% New pages with section.
%\renewcommand{\chapter}{\newpage\chapter}

\sisetup
{
    range-units=single,
    range-exponents=combine,
    range-phrase=--,
    inter-unit-product=\,,
}

% Efficiency symbols.
\newcommand{\effvmi}{E_v}
\newcommand{\effsvp}{E_p}
\newcommand{\efftri}{E_r}

% Supposedly gives 1.5 line spacing in Word.
\linespread{1.25}

% Adds 'repeated' onto tagged references.
\newcommand{\retag}[1]{\tag{\ref{#1} repeated}}

% Retagging of figures.
%  https://tex.stackexchange.com/questions/570165/using-tag-with-a-figure
\newcommand{\figureretag}[1]{%
  \renewcommand{\thefigure}{\ref{#1} (repeated)}%
  \addtocounter{figure}{-1}%
}
\newcommand{\subfigureretag}[1]{%
  \renewcommand{\thesubfigure}{Figure \ref{#1} (repeated)}%
  \addtocounter{subfigure}{-1}%
}

% Quotes in maths mode.
\DeclareMathSymbol{\mlq}{\mathord}{operators}{``}
\DeclareMathSymbol{\mrq}{\mathord}{operators}{`'}

% Theorems.
\newtheorem{theorem}{Theorem}

% For askterisk-ing things.
\NewDocumentCommand{\anote}{}{\makebox[0pt][l]{$^*$}}

% Stops broken links appearing over 2 pages. Can't imagine anyone would want a footnote over 2 pages anway.
\interfootnotelinepenalty=10000