\chapter{Conclusion} \label{sec:conclusions}    
    \section{Thesis summary}    
        % Overview.
        In this thesis, we have presented a mathematical model of maternal blood flow and oxygen concentration in the at-term human placenta, using a physiologically informed 2D organ-scale geometry. One third of stillbirths are related to placental dysfunction, and therefore the overarching motivation for this work has been to better understand the characteristics of diseased placentas, which will ultimately lead to improved pregnancy outcomes in the future.

        % Modelling flow + geometry.
        Modelling of the placenta is complicated by several factors including its development through gestation, its inherent multiscale nature, and the structural variability between placentas. Motivated by several previous studies, we considered maternal blood flow in the at-term human placenta, where we modelled the presence of the fetal villous tree as a rigid porous medium. Maternal blood flow was modelled using the Navier-Stokes-Darcy (NSD) equations, which captured the effects of free flow in areas such as the central cavity (CC) and the effects of porous flow in the intervillous space (IVS), with a physiologically sensible transition region between. We used these equations on a physically relevant geometry representing a 2D slice through a whole placenta, which captured structural features on a larger scale than previous studies; in particular, this geometry included six adjacent placentones that were partially separated by septal walls, as well as septal wall and marginal sinus veins. We then coupled the blood flow to a reaction-advection-diffusion equation that modelled oxygen transport, so that we could study the oxygen uptake by the fetal villous tree. As far as we are aware, this is the first time that a representative 2D whole-organ geometry has been used in the study of maternal flow.

        % Numerical methods + experiments (space).
        The approach of this thesis has been computational, where we have extensively used discontinuous Galerkin finite element methods (DGFEMs) to discretise spatial derivatives, which is advantageous in this application due to the method's simple treatment of complicated geometries, stability properties for large parameter variations, and favourable treatment of hyperbolic terms in PDEs. We believe this to be the first time that DGFEMs have been applied to the study of placental haemodynamics. We first used our DGFEM to compare the behaviour of the steady-state NSD flow field with related models currently used in the placental literature; here, flow in the IVS was observed to be similar across all models, with notable differences in behaviour in the CC. We also performed numerical experiments of blood flow and oxygen transport on the 2D placenta geometry with veins placed asymmetrically; we found that flow speed increased by an order of magnitude in the region closest to the chorionic plate, and oxygen concentration perfused radially from the spiral artery. This asymmetric experiment was important due to its physical relevance, and therefore formed the basis of the later MRI and utero-placental pump investigations.

        % Numerical methods + experiments (time).
        For time-dependent problems, we used a simple first-order backward Euler time-stepping scheme, which allowed us to briefly investigate the effects of pulsatile inflow on flow and oxygen transport; for our particular problem set-up, we found that pulsatile inflow had very little influence on flow and oxygen dynamics. We also derived a moving mesh DGFEM valid on moving domains, which allowed us to study the effects of a placental shape change on flow and oxygen dynamics. We verified our implementation of the numerical methods by achieving optimal spatial and temporal convergence rates in tests using the method of manufactured solution (MMS), and by comparing physical numerical experiments of placental flow with results in the existing placental literature.

        % Structure + variations.
        Whilst the overarching structure of the human placenta is well documented, contrasting estimates of the number and position of vessels in the experimental literature suggest that there is either a high variation between individual placentas or a lack of understanding. Furthermore, there are a number of conflicting hypotheses on where these vessels are located. This uncertainty has brought about two mathematical studies that have investigated the effect of vein placement on oxygen uptake \cite{chernyavskyMathematicalModelIntervillous2010,meklerImpactTissuePorosity2022}. We expanded upon this work by considering variations in the number and placement of both arteries and veins across our 2D placenta geometry, in addition to variations in other structural parameters such as the artery diameter and the IVS permeability. We used seven scalar-valued measures as a proxy for characterising the flow and oxygen concentration fields, thereby allowing us to investigate the impact of structural variations on placental function over several thousands of realisations. In summary, our results found the same `short-circuiting' effects that have been previously reported, that fewer veins drastically increased speeds across the placenta, and that the permeability of the IVS had by far the greatest effect on oxygen uptake. We also compared these results to previous experimental studies, allowing us to infer how structural changes may play a role in placental disease.

        % MRI.
        This thesis also presented a numerical MRI (or synthetic MRI) method for inferring sub-voxel velocity fields of in vivo placenta data. We introduced the basic physics underpinning signal measurement of motion-sensitising gradients in MRI scanning, along with an algorithm for computing signals of particles advected by fluid flow. We used this algorithm to compute MRI signals for a selection of simple manufactured flow fields and investigated the behaviour of signals in detail. Using an empirical signal decay model, we were able to identify a likely location for an artery in the in vivo data; the artery position allowed us to use MRI signals of (i) the simple manufactured flow, (ii) the simulated maternal flow, and (iii) the in vivo flow data, to infer a local flow field of the in vivo flow data.

        % Contractions.
        The recently-documented utero-placental pump is a contraction involving only the placenta, where placental volume can reduce by up to $40\%$ over a $10$-minute period, resulting in a periodic ejection of blood from the IVS. The mechanism of these contractions is currently unknown, which led us to develop the first preliminary model of this phenomenon, whereby we prescribed boundary motion as a first step to understand how shape change influences oxygen transport; we achieved this using a moving mesh DGFEM and a domain velocity that was inspired by MRI data. Our modelling showed that contractions of this form do have an impact on oxygen concentration dynamics and the oxygen uptake by the villous tree.

        % Novelties.
        To summarise, this thesis extends the existing literature in the following ways: (i) considers maternal flow on a physically relevant 2D whole-organ geometry, (ii) applies DGFEMs to study maternal flow in the placenta, (iii) investigates the effects of structural variations on placental function in great detail, (iv) provides a means of comparing computational flow fields with in vivo MRI data, and (v) develops the first preliminary model of the utero-placental pump as a first step in understanding how this phenomenon influences oxygen transport.
        
    \section{Future work}
        % \todoitemtwo{Something about parameter choices? See 2.4 summary. Also maybe don't bother.}
        \subsection{Model development}        
            % At term, maternal blood only.
            % NSD: could do Stokes; could vary permeability more; NSD transition width; Forcheimer.
            % Incompressibility and blood rheology [Bappoo, 2017].
            % Forcheimer term: The additional Forchheimer term is generally applicable to flow of high Reynolds number ($\Rey > 100$) \cite{nieldConvectionPorousMedia2017}.
            We adopted the incompressible Navier-Stokes-Darcy (NSD) equations for modelling maternal blood flow in the at-term placenta, which captured the effects of both free and porous flow through a spatially-varying permeability field. Our investigations in \S\ref{sec:numerical-methods:blood-flow-experiments:comparison} found that flow using this model gave similar flow in the IVS to two alternative flow models, with notable differences near the central cavity. The linear Stokes-Brinkman model presented in \S\ref{sec:modelling:blood-flow:s+b} may therefore prove sufficient in capturing the main flow features that take place outside the central cavity with a smaller computational cost. On the other hand, whilst simplifications of the model may be of value, the NSD blood flow model is the most general of the presented models, as it permits considerable flexibility in the specification of the spatially-varying permeability field. The choice of permeability field could be further exploited in future studies by, for example, using anatomical images to infer a permeability field that varies in regions of the IVS besides the central cavity and vessels.

            % Oxygen model isn't great; metabolism of placenta itself; other nutrients -- e.g. amino acids.
            % Future work: transport model is very simplistic: we're modelling oxygen as a disolved solute. And uptake by fetal vasculature proportional to concentration to blood is an extremely simplistic way of doing uptake.
            We used a reaction-advection-diffusion equation to model the transport of oxygen concentration dissolved in blood, where the advective velocity was taken from the blood flow model described above. This was a simple model which assumes that uptake by the fetal vasculature is proportional to the oxygen concentration in the maternal blood; future work could consider the additional oxygen binding dynamics that have been used by previous authors (e.g., \cite{serovOptimalVilliDensity2015, pearceImageBasedModelingBlood2016}), or consider the transport of other nutrients and waste products in the placenta.

            % Geometry simple; 3D; shape and size of vessels.
            The geometry we considered throughout this thesis was a highly simplified model of true placental structure, constructed by taking a 2D planar cross-section through a 3D spherical cap using approximations of placental dimensions. We made several simplifying assumptions, including that the cross-section intersects perfectly through the centre of all vessels, as well as an assumed shape and size of these vessels. An obvious extension to our work is to consider organ-scale maternal flow on a representative 3D placenta geometry, constructed either through a similar procedure to what is presented in 2D in this thesis, or from in vivo imaging data. Nevertheless, the 2D approach here has been essential in supporting the preliminary work of the University of Nottingham's Wellcome Leap In Utero project, SWiRL.
            
        \subsection{Numerical methods}
            % Higher-order time-stepping.
            We employed a discontinuous Galerkin finite element method (DGFEM) to discretise spatial derivatives, and a simple backward Euler scheme to discretise temporal derivatives. The first-order accuracy of the backward Euler scheme limited us to a choice of relatively small time-step sizes. An obvious improvement would be to select another unconditionally stable scheme such as the trapezoidal rule, backward differentiation formula (BDF), or a scheme from the Runge-Kutta family; these schemes would permit larger time-step sizes and would allow for longer time simulations.

            % Adaptivity. Current resolution: 2,335,705 elements, 35,035,575 velocity DoFs, 7,007,115 transport DoFs.
            The computational meshes used in this thesis comprised many different size elements, with a coarse mesh deeper into the IVS, and a finer mesh in the vessels and surrounding central cavity; in general, we found a relatively high mesh resolution was required to resolve spurious overshoots in the oxygen concentration problem, with the mesh used for the asymmetrical study in \S\ref{sec:numerical-methods:blood-flow-experiments:asymmetric} (and related studies through Chapters \ref{sec:numerical-mri} and \ref{sec:contractions}) consisting of \num{2335705} mesh elements. A simplification we made was that the computational mesh must remain the same between the blood flow and oxygen concentration discretisations; future work could relax this assumption, thereby using a coarse mesh for the blood flow approximation and a fine mesh for the oxygen transport approximation, which would ultimately lower the overall computational cost. Furthermore, techniques such as artificial viscosity or adaptivity could also be employed, allowing the use of meshes with fewer elements and thereby further lowering the computational cost.

        \subsection{Effects of placental structural on placental efficiency}
            % Eight measures may not be totally representative; e.g. height up septal wall; particle tracking.
            The approach of Chapter \ref{sec:nutrient-uptake} was to consider several thousands of realisations of flow and oxygen concentration fields, each characterised by seven lower-dimensional measures. There are likely several other choices of measure that would have given us further insight into the flow and oxygen transport behaviour, and therefore a better understanding of placental disease, such as measures characterising vessel position, or outlet routes via particle tracking.

            % Could study locations separately; Fixed positions of vessels is a strong assumption; Could study vein types separately (inc. turning off marginal sinuses).
            The first part of this chapter considered variations in both the number and position of vessels. These variations were considered simultaneously, and therefore an obvious next step would be to consider these effects separately. In addition, the geometry was parameterised with at most two basal plate veins per placentone, three septal wall veins, and two permanently retained marginal sinus veins. Future work could consider geometries without these restrictions, and also investigate the role that each type of vein has in blood flow and oxygen transport.

            % "Other" variations could benefit from the asymmetric flow field, in line with Chapters 5 and 6; Artery width didn't affect uptake -> damage to villous tree could be modelled (shear flow from Lecarpentier?); Variations chapter: variations to flow speed would likely have had a large effect.
            The second part of this chapter considered variations in seven other parameters, with fixed numbers and positions of vessels; this was an assumption, and subsequent work could investigate the different choices of vessel placement in combination with these other parameter variations. Our results most notably found that our study of variations in artery width had minimal impact on the measures, despite physiological reports that small artery widths are associated with disease (e.g., \cite{burtonRheologicalPhysiologicalConsequences2009}). We noted that this was likely due to damage to the villous tree, rather than behaviour that small artery widths directly influence, and therefore future work could consider a model of villous tree damage. Future work could also consider the effects of other likely impactful parameter variations, such as the cavity transition width and artery shape, and their influence on placental disease.

            % 2D vs 3D.
            Whilst the results of this thesis are restricted to 2D, the computational demand is much lower than an equivalent 3D study; this comprehensive 2D study has allowed us to focus the more computationally intensive investigations of \citeauthor{crowsonInvestigatingPlacentalHemodynamics2024} \cite{crowsonInvestigatingPlacentalHemodynamics2024}.

        \subsection{Numerical MRI}
            % Particle paths aren't streamlines; advecting outside voxels.
            Chapter \ref{sec:numerical-mri} introduced a method for numerical MRI, where signals are computed numerically for some underlying flow field. For computational simplicity, we made the simplification that each particle's velocity remained constant for the duration of the MRI pulse. Future work could therefore consider particle trajectories along the streamlines of the flow.

            % Real life is 3D MRI, so...; Comparing velocities without anatomical geometry isn't easy or robust.
            The approach of this chapter allowed us to infer a local flow field in the MRI data by manually selecting voxels with similar behaviour. However, there are inherent issues in locating comparable voxels between simulated and physical flow data without computing simulated flow fields on the same placental geometry. Future work could compute flow and numerical MRI signals on a comparable 3D geometry, or design an automatic algorithm that selects similar voxels in both the physical MRI and simulated MRI data for comparison.

            % Oxygen MRI; % Other organs?
            This thesis only considered motion-sensitising MRI sequences; an interesting extension would therefore be to compute MRI signals using our simulated oxygen concentration field, and compare this to in vivo MRI oxygenation data. Additionally, the majority of the presented methods are unspecific to the placenta, as they only require an underlying flow field. Therefore, future work could reapply these techniques to MRI of other organs such as the brain.

        \subsection{Utero-placental pump}
            % Mechanism completely ignored; assumed a terrible domain velocity; didn't even use the correct data, you spoon; directions of the contraction ignored; villous density unchanged, and this has a big impact on flow and oxygen; Reapply villous tree contractions? Muscular basal plate? Hypoxia?
            % As far as we are aware, Chapter \ref{sec:contractions} presented the first preliminary mathematical model of the utero-placental pump, which for simplicity neglected the mechanisms of contraction, and instead considered a representative volume change using a prescribed domain velocity. The form of the domain velocity assumed that contractions were uniform in both directions and evolved sinusoidally through time, which does not match experimental observations. Further work could investigate the appropriateness of contracting villous trees in describing the utero-placental pump, or some other biophysical mechanism that respects the data of \citeauthor{gowlandCharacterisingPlacentalContractions2024} \cite{gowlandCharacterisingPlacentalContractions2024}.

            As far as we are aware, Chapter \ref{sec:contractions} presented the first mathematical model of the utero-placental pump. For simplicity, we considered a representative volume change using a prescribed domain velocity. Future work taking this approach could investigate the effect of different choices of this domain velocity that better match physiological observations, such as a domain velocity that changes placental shape in only the horizontal direction.
            
            Furthermore, an interesting extension to this work would be to consider the appropriateness of contracting villous trees in describing the utero-placental pump, for instance by reapplying the work of \citeauthor{katoVillousTreeModel2017} \cite{katoVillousTreeModel2017}. Alternatively, other biophysical mechanisms could be investigated which respect the data of \citeauthor{gowlandCharacterisingPlacentalContractions2024} \cite{gowlandCharacterisingPlacentalContractions2024}.
            
            % The remaining material in this thesis is as follows. Appendix \ref{sec:smooth-transition} formally defines the smooth transition function $\Psi$, used to denote the presence of fetal villous tree in Chapter \ref{sec:modelling}. Appendix \ref{sec:flow-comparison} expands upon the results of \S\ref{sec:numerical-methods:blood-flow-experiments:comparison} between the four different flow models by computing differences between the flow fields. Appendix \ref{sec:lax-friedrichs} derives the $\alpha$ parameter used in the modified Lax-Friedrichs flux of \S\ref{sec:contractions:dgfem-discretisation}. This thesis ends with the list of bibliographic references.